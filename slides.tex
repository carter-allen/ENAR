\PassOptionsToPackage{unicode=true}{hyperref} % options for packages loaded elsewhere
\PassOptionsToPackage{hyphens}{url}
%
\documentclass[ignorenonframetext,]{beamer}
\usepackage{pgfpages}
\setbeamertemplate{caption}[numbered]
\setbeamertemplate{caption label separator}{: }
\setbeamercolor{caption name}{fg=normal text.fg}
\beamertemplatenavigationsymbolsempty
% Prevent slide breaks in the middle of a paragraph:
\widowpenalties 1 10000
\raggedbottom
\setbeamertemplate{part page}{
\centering
\begin{beamercolorbox}[sep=16pt,center]{part title}
  \usebeamerfont{part title}\insertpart\par
\end{beamercolorbox}
}
\setbeamertemplate{section page}{
\centering
\begin{beamercolorbox}[sep=12pt,center]{part title}
  \usebeamerfont{section title}\insertsection\par
\end{beamercolorbox}
}
\setbeamertemplate{subsection page}{
\centering
\begin{beamercolorbox}[sep=8pt,center]{part title}
  \usebeamerfont{subsection title}\insertsubsection\par
\end{beamercolorbox}
}
\AtBeginPart{
  \frame{\partpage}
}
\AtBeginSection{
  \ifbibliography
  \else
    \frame{\sectionpage}
  \fi
}
\AtBeginSubsection{
  \frame{\subsectionpage}
}
\usepackage{lmodern}
\usepackage{amssymb,amsmath}
\usepackage{ifxetex,ifluatex}
\usepackage{fixltx2e} % provides \textsubscript
\ifnum 0\ifxetex 1\fi\ifluatex 1\fi=0 % if pdftex
  \usepackage[T1]{fontenc}
  \usepackage[utf8]{inputenc}
  \usepackage{textcomp} % provides euro and other symbols
\else % if luatex or xelatex
  \usepackage{unicode-math}
  \defaultfontfeatures{Ligatures=TeX,Scale=MatchLowercase}
\fi
% use upquote if available, for straight quotes in verbatim environments
\IfFileExists{upquote.sty}{\usepackage{upquote}}{}
% use microtype if available
\IfFileExists{microtype.sty}{%
\usepackage[]{microtype}
\UseMicrotypeSet[protrusion]{basicmath} % disable protrusion for tt fonts
}{}
\IfFileExists{parskip.sty}{%
\usepackage{parskip}
}{% else
\setlength{\parindent}{0pt}
\setlength{\parskip}{6pt plus 2pt minus 1pt}
}
\usepackage{hyperref}
\hypersetup{
            pdftitle={Talk},
            pdfauthor={Carter Allen; Brian Neelon; Sara E. Benjamin-Neelon},
            pdfborder={0 0 0},
            breaklinks=true}
\urlstyle{same}  % don't use monospace font for urls
\newif\ifbibliography
\setlength{\emergencystretch}{3em}  % prevent overfull lines
\providecommand{\tightlist}{%
  \setlength{\itemsep}{0pt}\setlength{\parskip}{0pt}}
\setcounter{secnumdepth}{0}

% set default figure placement to htbp
\makeatletter
\def\fps@figure{htbp}
\makeatother


\title{Talk}
\author{Carter Allen \and Brian Neelon \and Sara E. Benjamin-Neelon}
\date{}

\begin{document}
\frame{\titlepage}

\begin{frame}{Introduction: Infant Motor Development}
\protect\hypertarget{introduction-infant-motor-development}{}

\begin{itemize}
\item
  Infant motor development has been shown to be an important predictor
  of health later in life.
\item
  Early motor development has been associated with improved physical
  activity, cognitive function, and educational attainment.
\item
  Delayed motor development has been linked to congnitive disorders in
  adulthood.
\item
  \textbf{Interest lies in identifying factors related to differing
  developmental patterns.}
\end{itemize}

\end{frame}

\begin{frame}{Nurture Study}
\protect\hypertarget{nurture-study}{}

\begin{itemize}
\item
  Birth cohort of predominantly black women and their infants residing
  in the southeastern U.S. between 2013 and 2017
\item
  Infant development was assessed quarterly through the first year of
  life at 3, 6, 9, and 12 months of age.
\item
  \textbf{Bayley composite score} was used to measure infant
  development. Scores range from 40 to 160, with higher scores
  indicating more advanced development.
\item
  \textbf{Food security} was assessed during pregnancy and infancy. It
  was hypothesized that \emph{lack of food security during pregnancy
  contributes to slower motor development during infancy}.
\end{itemize}

\end{frame}

\begin{frame}{Statistical Challenges}
\protect\hypertarget{statistical-challenges}{}

\begin{itemize}
\item
  Standard repeated measures models of Bayley scores in the Nurture data
  feature \textbf{skewed residuals}.
\item
  Bayley scores exhibit \textbf{intermittent missingness}, with
  approximately one third of Bayley scores missing at each timepoint.
\item
  Bayley scores are correlated through time in an \textbf{unstructured
  manner}.
\item
  Existing approaches to modeling development clusters fail to account
  for one of more of these relevant features of the Nurture data.
\end{itemize}

\end{frame}

\begin{frame}{Model: Mixture Model}
\protect\hypertarget{model-mixture-model}{}

We propose a finite mixture model that accommodates skewness in
residuals, non-ignorable missingness in outcomes, and unstructured
dependence among repeated measures.

The model is based on the multivariate skew normal (MSN) distribution,
with extensions to the multivariate skew-\(t\) (MST) distribution.

Specifically, let \(\mathbf{y}_{i}=(y_{i1},\ldots,y_{iJ})^T\) be a
\(J \times 1\) vector of standardized Bayley scores for subject
\(i\;(i=1,\ldots,n)\). We propose a mixture model of the form
\begin{equation}
    \label{eq:mixture}
f(\mathbf{y}_i) = \sum_{k = 1}^{K} \pi_{ki} f(\mathbf{y}_i|\boldsymbol\theta_k),
\end{equation} where \(\boldsymbol\theta_k\) is the set of parameters
specific to cluster \(k\) (\(k = 1,...,K\)) and \(\pi_{ki}\) is a
subject-specific mixing weight representing the probability that subject
\(i\) belongs to cluster \(k\).

\end{frame}

\begin{frame}{Model: MSN Components}
\protect\hypertarget{model-msn-components}{}

For posterior inference, we introduce a latent cluster indicator
variable \(z_i\) taking the value \(k \in \{1,...,K\}\) with probability
\(\pi_{ki}\). Given \(z_i = k\), we assume \(\mathbf{y}_{i}\) is
distributed according to a \(J\)-dimensional multivariate skew normal
(MSN) density: \begin{equation}
\mathbf{y}_{i}|(z_i=k) &\stackrel{ind}{\sim}& \text{MSN}_J(\boldsymbol\zeta_{ki},\boldsymbol\alpha_k,\boldsymbol\Omega_k), ~\text{with density} \label{eq:msndens}\nonumber
\end{equation}

\end{frame}

\end{document}
